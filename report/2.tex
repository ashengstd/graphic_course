\documentclass[a4paper,12pt]{article}
\usepackage{geometry}
\usepackage{ctex}
\usepackage{fontspec}
\usepackage{graphicx}
\geometry{left=3cm, right=3cm, top=3cm, bottom=3cm}

\usepackage{minted}
\setminted{linenos=false, breaklines=true}
\usemintedstyle{one-dark}
\usepackage{mdframed}
\surroundwithmdframed{minted}

\begin{document}

% 封面内容的整体居中
\begin{titlepage}
    \centering
    \vspace*{3cm} % 调整顶部留白

    % 标题部分
    {\Huge \textbf{计算机图形学程序设计}\par}
    {\Huge \textbf{边框\&线填充}\par}
    \vspace{0.5cm}
    {\Huge \textbf{实验报告}\par}

    \vspace{8cm} % 调整中间留白

    % 学号和姓名部分
    {\Large
    \begin{tabular}{rl}
        学号: & 08212012 \\
        姓名: & 任笙
    \end{tabular}
    \par}

    \vfill % 调整底部留白,自动对齐到底部
\end{titlepage}

\section{实验目的}
本实验的目的是通过实现 \textbf{扫描线填充算法 (Scanline Fill Algorithm)},深入理解计算机图形学中多边形填充的基本原理和实现方法。具体目标包括:
\begin{enumerate}
    \item 学习扫描线算法的工作流程,包括边表和活动边表的构建与更新。
    \item 实现多边形的边框绘制及其内部填充。
    \item 探讨边框像素的处理以及填充效率优化。
\end{enumerate}

\section{实验代码}
\begin{minted}{cpp}
void drawLine(int x1, int y1, int x2, int y2) {
  int dx = x2 - x1;
  int dy = y2 - y1;

  int steps = max(abs(dx), abs(dy));
  float xIncrement = dx / (float)steps;
  float yIncrement = dy / (float)steps;

  float x = x1;
  float y = y1;

  for (int i = 0; i <= steps; i++) {
    setPixel(round(x), round(y));
    x += xIncrement;
    y += yIncrement;
  }
}
void ScanlineFill() {
  vector<Edge> edgeTable[maxHt];
  int numVertices = LINE_NUM;

  // 构建边表
  for (int i = 0; i < numVertices; i++) {
    int x1 = vertices[4 * i];
    int y1 = vertices[4 * i + 1];
    int x2 = vertices[4 * i + 2];
    int y2 = vertices[4 * i + 3];

    if (y1 == y2)
      continue; // 跳过水平边

    if (y1 > y2) {
      swap(x1, x2);
      swap(y1, y2);
    }

    Edge edge;
    edge.yMax = y2;
    edge.xMin = x1;
    edge.slopeInverse = (float)(x2 - x1) / (y2 - y1);
    edgeTable[y1].push_back(edge);
  }

  vector<Edge> activeEdgeTable;

  // 扫描线填充
  for (int y = 0; y < maxHt; y++) {
    // 更新AET
    for (auto it = activeEdgeTable.begin(); it != activeEdgeTable.end();) {
      if (it->yMax == y) {
        it = activeEdgeTable.erase(it);
      } else {
        it->xMin += it->slopeInverse;
        ++it;
      }
    }

    // 将新的边加入AET
    activeEdgeTable.insert(activeEdgeTable.end(), edgeTable[y].begin(),
                           edgeTable[y].end());

    // 按xMin排序
    sort(activeEdgeTable.begin(), activeEdgeTable.end(),
         [](Edge a, Edge b) { return a.xMin < b.xMin; });

    // 填充像素
    for (size_t i = 0; i < activeEdgeTable.size(); i += 2) {
      if (i + 1 >= activeEdgeTable.size())
        break;

      int xStart = ceil(activeEdgeTable[i].xMin);
      int xEnd = floor(activeEdgeTable[i + 1].xMin);

      // 跳过边框像素
      for (int x = xStart; x <= xEnd; x++) {
        // 边框检查:防止覆盖边框
        bool isOnBorder = (x == round(activeEdgeTable[i].xMin)) ||
                          (x == round(activeEdgeTable[i + 1].xMin));
        if (!isOnBorder) {
          setPixel(x, y);
        }
      }
    }
  }
}
\end{minted}
\section{实验内容}
本实验使用了以下关键方法:
\begin{itemize}
    \item 使用 \textbf{DDA 算法} 绘制多边形的边框。
    \item 构建边表(Edge Table),将多边形的边按照起点的 $y$ 坐标组织并存储其最大 $y$ 值、最小 $x$ 值以及斜率倒数。
    \item 使用扫描线逐行填充多边形内部像素,并通过活动边表(Active Edge Table, AET)动态维护当前扫描线的交点。
    \item 跳过边框像素,确保填充过程中边框清晰可见。
\end{itemize}

\section{实验结果}
\subsection*{1. 边框绘制}
程序成功使用 DDA 算法绘制了多边形的边框,顶点序列如下:
\[
\{(200, 100), (400, 120), (480, 300), (300, 400), (120, 300), (200, 100)\}
\]

\subsection*{2. 内部填充}
\begin{itemize}
    \item 填充算法准确计算了多边形每一条扫描线与边框的交点。
    \item 填充颜色与边框颜色不同(边框为绿色,填充为白色),实现了可视化效果的区分。
    \item 边框像素未被覆盖,保留了清晰的轮廓。
\end{itemize}

\subsection*{3. 执行效率}
在测试中,算法能够实时填充分辨率为 $500 \times 500$ 的窗口,展示了扫描线填充算法的高效性。

\section*{总结}
本实验通过实现扫描线填充算法,加深了对多边形填充机制的理解。扫描线算法通过使用边表和活动边表,成功实现了复杂多边形的快速填充。未来可以在以下方面优化:
\begin{itemize}
    \item 处理凹多边形的填充,以及特殊情况下的边界处理。
    \item 支持颜色渐变填充,增强视觉效果。
    \item 在更高分辨率下测试算法性能,并尝试硬件加速实现。
\end{itemize}

\begin{figure}
    \centering
    \includegraphics[width=0.75\linewidth]{tianchong.png}
    \caption{Result1}
    \label{fig:enter-label}
\end{figure}

\end{document}
