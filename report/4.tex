\documentclass[a4paper,12pt]{article}
\usepackage{geometry}
\usepackage{ctex}
\usepackage{fontspec}
\usepackage{graphicx}
\geometry{left=3cm, right=3cm, top=3cm, bottom=3cm}
\usepackage{amsmath}
\usepackage{minted}
\setminted{linenos=false, breaklines=true}
\usemintedstyle{one-dark}
\usepackage{mdframed}
\surroundwithmdframed{minted}

\begin{document}

% 封面内容的整体居中
\begin{titlepage}
    \centering
    \vspace*{3cm} % 调整顶部留白

    % 标题部分
    {\Huge \textbf{计算机图形学程序设计}\par}
    {\Huge \textbf{图像旋转}\par}
    \vspace{0.5cm}
    {\Huge \textbf{实验报告}\par}

    \vspace{8cm} % 调整中间留白

    % 学号和姓名部分
    {\Large
    \begin{tabular}{rl}
        学号: & 08212012 \\
        姓名: & 任笙
    \end{tabular}
    \par}

    \vfill % 调整底部留白,自动对齐到底部
\end{titlepage}

\section{实验目的}
本实验的主要目的是实现一个基于Eigen和OpenCV的简单光栅化渲染器,完成以下功能:
\begin{itemize}
    \item 实现视图变换、模型变换和投影变换。
    \item 支持绕任意轴的旋转操作。
    \item 验证光栅化渲染器在生成2D图像中的表现。
\end{itemize}


\section{实验代码}
\begin{minted}{python}
Eigen::Matrix4f get_model_matrix(float rotation_angle) {
  Eigen::Matrix4f model = Eigen::Matrix4f::Identity();

  // 将角度转换为弧度
  float radian = rotation_angle * MY_PI / 180.0;

  // 构造绕 Z 轴的旋转矩阵
  model << cos(radian), -sin(radian), 0, 0, sin(radian), cos(radian), 0, 0, 0,
      0, 1, 0, 0, 0, 0, 1;

  return model;
}

Eigen::Matrix4f get_projection_matrix(float eye_fov, float aspect_ratio,
                                      float zNear, float zFar) {
  Eigen::Matrix4f projection = Eigen::Matrix4f::Identity();
  float angel = eye_fov / 180.0 * MY_PI;
  float t = zNear * std::tan(angel / 2);
  float r = t * aspect_ratio;
  float l = -r;
  float b = -t;

  Eigen::Matrix4f MorthoScale(4, 4);
  MorthoScale << 2 / (r - l), 0, 0, 0, 0, 2 / (t - b), 0, 0, 0, 0,
      2 / (zFar - zNear), 0, 0, 0, 0, 1;

  Eigen::Matrix4f MorthoPos(4, 4);
  MorthoPos << 1, 0, 0, -(r + l) / 2, 0, 1, 0, -(t + b) / 2, 0, 0, 1,
      -(zNear + zFar) / 2, 0, 0, 0, 1;

  Eigen::Matrix4f Mpersp2ortho(4, 4);

  Mpersp2ortho << zNear, 0, 0, 0, 0, zNear, 0, 0, 0, 0, zNear + zFar,
      -zNear * zFar, 0, 0, 1, 0;

  Eigen::Matrix4f Mt(4, 4);
  Mt << 1, 0, 0, 0, 0, 1, 0, 0, 0, 0, -1, 0, 0, 0, 0, 1;
  Mpersp2ortho = Mpersp2ortho * Mt;

  projection = MorthoScale * MorthoPos * Mpersp2ortho * projection;

  return projection;

  // Edit end
}

Eigen::Matrix4f get_rotation(Eigen::Vector3f axis, float angle) {
  Eigen::Matrix4f rotation = Eigen::Matrix4f::Identity();

  // 确保轴为单位向量
  Eigen::Vector3f u = axis.normalized();

  float cos_theta = std::cos(angle * MY_PI / 180.0);
  float sin_theta = std::sin(angle * MY_PI / 180.0);

  // 构造 [u]_x 矩阵
  Eigen::Matrix3f ux;
  ux << 0, -u.z(), u.y(), u.z(), 0, -u.x(), -u.y(), u.x(), 0;

  // 构造 u * u^T
  Eigen::Matrix3f uuT = u * u.transpose();

  // 计算旋转矩阵
  Eigen::Matrix3f R = cos_theta * Eigen::Matrix3f::Identity() +
                      (1 - cos_theta) * uuT + sin_theta * ux;

  // 将 3x3 旋转矩阵嵌入到 4x4 矩阵中
  rotation.block<3, 3>(0, 0) = R;

  return rotation;
}
\end{minted}

\section{实验结果}
通过实验,我们成功实现了各项变换功能,并基于简单的三角形顶点数据生成了对应的渲染图像。以下是主要实验结果和代码中的实现细节:

\subsection*{视图变换}
视图矩阵通过平移操作将观察者视点移动到世界坐标系原点。实现方法如下:
\[
\text{View Matrix} = 
\begin{bmatrix}
1 & 0 & 0 & -x \\
0 & 1 & 0 & -y \\
0 & 0 & 1 & -z \\
0 & 0 & 0 & 1 \\
\end{bmatrix}
\]

\subsection*{模型变换}
模型变换支持绕任意轴旋转,通过Rodrigues公式实现。公式如下:
\[
\mathbf{R} = \cos\theta \mathbf{I} + (1-\cos\theta) \mathbf{u}\mathbf{u}^T + \sin\theta [\mathbf{u}]_x
\]
其中:
\begin{itemize}
    \item \(\mathbf{u}\) 是单位旋转轴向量。
    \item \([\mathbf{u}]_x\) 是\(\mathbf{u}\)的叉乘矩阵。
\end{itemize}

\subsection*{投影变换}
投影变换实现了从透视投影到正交投影的映射,并通过比例缩放和平移操作完成最终的正交投影。矩阵定义如下:
\[
\text{Projection Matrix} = M_\text{ortho-scale} \cdot M_\text{ortho-pos} \cdot M_\text{persp-to-ortho}
\]

\subsection*{渲染结果}
实验中使用了简单的三角形数据,生成了以下光栅化图像:
\begin{itemize}
    \item 初始角度为0°时的渲染结果。
    \item 随用户输入旋转角度变化的动态渲染结果。
\end{itemize}
\begin{figure}
    \centering
    \includegraphics[width=0.75\linewidth]{0_degree.png}
    \caption{0°的渲染结果}
    \label{fig:0_degree}
\end{figure}
\begin{figure}
    \centering
    \includegraphics[width=0.75\linewidth]{var.png}
    \caption{角度旋转的结果}
    \label{fig:var}
\end{figure}
\section*{总结}
本实验完成了光栅化渲染器的基本功能实现,包括视图变换、模型变换和投影变换。代码中采用了Eigen库进行矩阵运算,并使用OpenCV实现图像输出。实验结果表明,该渲染器能够正确处理旋转变换及投影变换。未来工作可扩展至更复杂的几何体渲染及着色技术的引入。


\end{document}
