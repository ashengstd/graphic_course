\documentclass[a4paper,12pt]{article}
\usepackage{geometry}
\usepackage{ctex}
\usepackage{fontspec}
\usepackage{graphicx}
\geometry{left=3cm, right=3cm, top=3cm, bottom=3cm}

\usepackage{minted}
\setminted{linenos=false, breaklines=true}
\usemintedstyle{one-dark}
\usepackage{mdframed}
\surroundwithmdframed{minted}

\begin{document}

% 封面内容的整体居中
\begin{titlepage}
    \centering
    \vspace*{3cm} % 调整顶部留白

    % 标题部分
    {\Huge \textbf{计算机图形学程序设计}\par}
    \vspace{0.5cm}
    {\Huge \textbf{实验报告}\par}

    \vspace{8cm} % 调整中间留白

    % 学号和姓名部分
    {\Large
    \begin{tabular}{rl}
        学号: & 08212012 \\
        姓名: & 任笙
    \end{tabular}
    \par}

    \vfill % 调整底部留白,自动对齐到底部
\end{titlepage}

\section{实验目的}


本实验旨在通过编程实现和可视化 \textbf{Bezier 曲线},以加深对其数学原理及应用的理解。具体目标包括:
\begin{enumerate}
    \item 学习 \textbf{Bezier 曲线} 的定义及其通过组合数和 Bernstein 多项式生成的方法。
    \item 使用分段的方式绘制高阶和低阶 Bezier 曲线,分析曲线的平滑性和连续性。
    \item 探索控制点对 Bezier 曲线形状的影响,并通过程序输出 Bezier 曲线的函数表达式,展示控制点在曲线生成中的作用。
\end{enumerate}

\section{实验代码}
\begin{minted}{cpp}
// 计算阶乘
int factorial(int n) {
  if (n <= 1)
    return 1;
  return n * factorial(n - 1);
}

// 计算组合数 C(n, i)
int combination(int n, int i) {
  return factorial(n) / (factorial(i) * factorial(n - i));
}

// 计算 Bezier 曲线点
Point bezierPoint(const vector<Point> &points, float t) {
  int n = points.size() - 1;
  Point p(0, 0);
  for (int i = 0; i <= n; i++) {
    float coef = combination(n, i) * pow(1 - t, n - i) * pow(t, i);
    p.x += coef * points[i].x;
    p.y += coef * points[i].y;
  }
  return p;
}
\end{minted}
\section{实验结果}
\subsection*{1. Bezier 曲线的函数表达式}
实验程序通过数学公式计算并输出了控制点生成的 6 次 Bezier 曲线函数表达式,以及分段 3 次 Bezier 曲线的函数表达式。示例如下:
\begin{itemize}
    \item 6 次 Bezier 曲线(包含 7 个控制点):
    \[
    x(t) = \sum_{i=0}^6 C(6, i) \cdot P_i.x \cdot (1-t)^{6-i} \cdot t^i
    \]
    \[
    y(t) = \sum_{i=0}^6 C(6, i) \cdot P_i.y \cdot (1-t)^{6-i} \cdot t^i
    \]
    \item 分段 3 次 Bezier 曲线分别由前 4 个控制点和后 4 个控制点生成,其表达式类似,但次数为 3。
\end{itemize}

\subsection*{2. 可视化结果}
\begin{itemize}
    \item 在窗口中绘制了一条 \textbf{6 次 Bezier 曲线},并展示了控制点的分布。
    \item 曲线被分为两个 3 次 Bezier 曲线段,用不同颜色区分。
    \item 整体曲线平滑且连续,且分段曲线间具备 1 阶连续性。
\end{itemize}

\subsection*{3. 控制点的影响}
实验中,调整控制点的数量和位置可以显著改变曲线的形状,验证了控制点对 Bezier 曲线生成的决定性作用。

\section*{总结}
通过实验,我实现了 \textbf{Bezier 曲线} 的数学建模、表达式生成及其分段实现,验证了高阶曲线与低阶曲线的区别,并直观理解了控制点的作用。本实验加深了对计算机图形学中曲线生成理论的认识,并为后续复杂图形绘制提供了技术支持。

\begin{figure}
    \centering
    \includegraphics[width=0.75\linewidth]{image.png}
    \caption{Result}
    \label{fig:enter-label}
\end{figure}

\end{document}
